\documentclass[english]{article}
\usepackage{booktabs}
\usepackage{babel}
\usepackage{varioref}
% Hyperlinks im Dokument
\usepackage[
  english,
  unicode,        % Unicode in PDF-Attributen erlauben
  pdfusetitle,    % Titel, Autoren und Datum als PDF-Attribute
  pdfcreator={},  % ┐ PDF-Attribute säubern
  pdfproducer={}, % ┘
]{hyperref}
% erweiterte Bookmarks im PDF
\usepackage{bookmark}
% Captions schöner machen.
\usepackage[
  labelfont=bf,        % Tabelle x: Abbildung y: ist jetzt fett
  font=small,          % Schrift etwas kleiner als Dokument
  width=0.9\textwidth, % maximale Breite einer Caption schmaler
]{caption}
% subfigure, subtable, subref
\usepackage{subcaption}
% Grafiken können eingebunden werden
\usepackage{graphicx}

% schöne Tabellen
\usepackage{booktabs}

% Verbesserungen am Schriftbild
\usepackage{microtype}

\usepackage[margin=2cm]{geometry}
\usepackage{amsmath, amssymb}
\usepackage{hyperref}
\usepackage{graphicx}
\usepackage{float}
\usepackage{aligned-overset}
\usepackage{subcaption}
\usepackage{tabularx} % fuer gleichungen nebeneinander
\usepackage{wrapfig} % damit figures neben text sien koennen

% partielle ableitungen
\newcommand{\delr}{\partial_r}
\newcommand{\delt}{\partial_t}
\newcommand{\deltheta}{\partial_\theta}
\newcommand{\delphi}{\partial_\varphi}

% elektrische feldkonstante
\newcommand{\epsz}{\epsilon_0}
% 1 / 4pi eps
\newcommand{\kco}{\frac{1}{4\pi\epsilon_0}}
% del fuer partial
\newcommand{\del}{\partial}

% div und rot
\newcommand{\diver}{\vec \nabla \cdot}
\newcommand{\rot}{\vec \nabla \times}

% hyperbolische funktionen
\newcommand{\arsinh}{\text{arsinh}}
\newcommand{\arcosh}{\text{arcosh}}
\newcommand{\artanh}{\text{artanh}}

% fuer impedanzen 
\newcommand{\omegaC}{\omega C}
\newcommand{\omegaL}{\omega L}
\newcommand{\omegaR}{\omega R}
\newcommand{\omegaLC}{\omega LC}
\newcommand{\omegaLCR}{\omega LCR}
\newcommand{\omegaCR}{\omega CR}

% brackets
\usepackage{mathtools}
\DeclarePairedDelimiter\bra{\langle}{\rvert}
\DeclarePairedDelimiter\ket{\lvert}{\rangle}
\DeclarePairedDelimiterX\braket[2]{\langle}{\rangle}{#1 \delimsize\vert #2}

% fancy header
\usepackage{fancyhdr}
\fancyhf{}
% vspaces in den headern fuer Distanzen notwendig
% linke Seite: Namen der Abgabegruppe
\lhead{\textbf{Quantum Information 1}\vspace{1.5cm}}
% rechte Seite: Modul, Gruppe, Semester
\rhead{\textbf{Matthias Maile}\vspace{1.5cm}}
% Center: nr. des blattes
\chead{\vspace{2.5cm}\huge{\textbf{1st Assignment}}}
% benoetigt damit der eigentliche Text nicht in der Überschrift steckt
\setlength{\headheight}{4cm}

% zum zeichnen tikz
\usepackage{tikz}

% fuer fabigen text
\usepackage{xcolor}

% irgendwas mit figures
\usepackage{subcaption}

% easier partial derivatives
\usepackage{derivative}

\setlength{\parskip}{0cm}
\setlength{\parindent}{0cm}

\newcommand*\Laplace{\mathop{}\!\mathbin\bigtriangleup}
\newcommand*\DAlambert{\mathop{}\!\mathbin\Box}


%\renewcommand{\thesubsection}{\alph{subsection})}


\begin{document}
\thispagestyle{fancy}

\subsubsection*{1. What is “Quantum Information”?}

The term “Quantum Information” can be discussed by looking at the two parts, 
“Quantum” and “Information”, so I will start by intrucing what “normal” information 
(or non-quantum information) is. \\
Information can be defined as  a quantified surprise of the 
outcome of a random event. As an example, if we throw a dice, we have no knowledge of 
the outcome (each number 1-6 has the chance of 1/6). However, by throwing the dice, we 
know the result and hereby gained information. A question that might arise is how we 
can measure the amount of information gained through this observation. One typical measure 
is known as the entropy
\[
	S = -\sum_{i} p_i \log(p_i),
\]
which for our example is $S = \log(6)$ (the base of the logarithm changes with different
definitions). This definition is really advantagous, since we no longer need to rely on
our intuition about information, but can easily calculate that throwing a dice yields
more information than a coin toss or even how much information can be stored in a 4MP
image.
\\
Now what makes quantum information special? The prefix quantum refers to the minimal 
amount of a physical entity, however, here the term can be expanded to quantum-mechanical 
information, or the information of a quantum mechanical system. So what is the quantum
mechanical equivalent to knowing the state of a system like with a coin toss? This can
depend on the system, some examples are
\begin{itemize}
	\item the energy level $E_n$ in a harmonic oscillator,
	\item orbital angular momentum or similar behaving quantum numbers 
		(e.g. $\hat L^2$ and $\hat L_z$, $\hat S^2$ and $\hat S_z$),
	\item the state of a n-state system.
\end{itemize}
But all of these are just ways to describe the wave function. As it can be seen in
\autoref{tab:wave_functions}, all these quantum numbers relate to a specific wave function,
which carries all the
information that is stored in a quantum mechanical system, which means that \emph{the wave
function is quantum information}.
\begin{table}[H]
  \centering
  \caption{Information stored in quantum numbers and the corresponding wave functions.}
  \label{tab:wave_functions}
  \begin{tabular}{ccc}
  \toprule
  Information about the System &
  Quantum number &
  Wavefunction \\
  \midrule
  Energy level in a harmonic oscillator & 
  $E_n$ & $\psi_n(x) \propto H_n(\tilde{x}) \cdot e^{-\tilde{x}^2}, \, \tilde{x} =
  \sqrt{\frac{m\omega}{\hbar}} x $ \\
  % angular momentum
  Angular momentum &
  $l^2 = \hat L^2$, $m_l = \hat L_z$ &
  Spherical harmonics: $\psi_{lm} \propto Y_l^m(\theta, \varphi)$ \\
  % Qubit
  State of a cubit &
  $a, b$ &
  $\ket{\psi} = a \ket{0} + b \ket{1}$ \\
  \end{tabular}
\end{table}

\newpage
\subsubsection*{2. Why do you want to study quantum information science?}
My interest in quantum information science stems from two aspects: The theoretical
background and it`s applications where quantum information science can be used to solve
problems.
\\
First I want to address the theoretical background. While I enjoy learning about
phenomena and experiments in physics, what I love even
more is seeing how we can use math to explain how the nature behaves. One area where this
is really noticable is quantum mechanics. Understanding how the state of a system can 
be represented by a vector in a Hilbert space and seeing it`s time evolution as a unitary
transformation catched my interest a lot more than the experiments in my quantum mechanics
class. Seeing how we can use those rather abstract concepts to store and transport data
fascinates me, which is why I am taking this course.
\\
In my last part I concluded that the  theoretical concepts can be used for data processing
and that is also where my second motivation for quantum information lies. With Moore`s law
coming to a limit it is becomming clear that we need to develop a new technology to build
computers. While a normal transistor based computer also uses quantum mechanical effects
to operate, using the concepts of quantum information theory to create a new way to
do computations would be really innovative. Not only the hardware itself would be new,
a qubit based computer is able to run algorithms specifically designed with quantum
information in mind, so called quantum algorithms. One famous example is Shor`s algorithm,
where a quantum computer is able to factor an integer in relatively short time, a problem
which a classical computer can only solve for small numbers.

\end{document}
